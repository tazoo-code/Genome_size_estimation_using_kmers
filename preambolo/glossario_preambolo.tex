
\usepackage[toc]{glossaries}

\makenoidxglossaries

\setacronymstyle{long-short-desc}

\newglossaryentry{snd}{
	text={distribuzione normale asimmetrica},
	name={Distribuzione normale asimmetrica}, 
	description={Distribuzione continua di probabilità che generalizza la distribuzione normale, permettendo un indice di asimmetria diverso da zero}
}

\newglossaryentry{dbn}{
	text={distribuzione binomiale negativa},
	name={Distribuzione binomiale negativa}, 
	description={Distribuzione discreta di probabilità di parametri $r$ e $p$, che descrive il numero di fallimenti in una sequenza di esperimenti indipendenti, ciascuno con probabilità di successo $p$, prima dell'occorrenza di un numero $r$ di successi},
	plural={distribuzioni binomiali negative}
}

\newglossaryentry{dp}{
	text={distribuzione di Poisson},
	name={Distribuzione di Poisson}, 
	description={Distribuzione discreta di probabilità che esprime la probabilità di un numero dato di eventi che si verificano successivamente e in modo indipendente, fissato un numero medio di occorrenze},
	plural={distribuzioni di Poisson}
}

\newglossaryentry{db}{
	text={distribuzione binomiale},
	name={Distribuzione binomiale}, 
	description={Distribuzione discreta di probabilità di parametri $n$ e $p$, che descrive il numero di successi in una sequenza di $n$ esperimenti indipendenti, ciascuno con probabilità di successo $p$},
	plural={distribuzioni binomiali}
}

\newglossaryentry{locus}{
	text={locus},
	name={Locus}, 
	description={Posizione specifica e fissata all'interno di un cromosoma in cui è posizionato un particolare gene o un'altra sequenza significativa},
	plural={loci}
}

\newglossaryentry{rate_eterozigosity}{
	text={rapporto di eterozigosi},
	name={Rapporto di eterozigosi}, 
	description={Rapporto tra il numero di siti eterozigoti e quello di siti omozigoti nel genoma di un individuo},
	plural={rapporti di eterozigosi}
}

\newglossaryentry{fitting}{
	text={fitting},
	name={Fitting}, 
	description={Costruzione di una curva o di una funzione matematica che abbia la migliore corrispondenza ad una serie di punti. Essa può implicare l'interpolazione, nel caso si richieda un'esatta corrispondenza tra i punti e la funzione da ricercare},
	plural={fitting}
}

\newacronym
	[description={Variazione della sequenza di DNA a carico di un singolo nucleotide, presente almeno nell'1\% degli individui di una popolazione}, longplural={polimorfismi a singolo nucleotide}, name={SNP - Polimorfismo a singolo nucleotide}, shortplural={SNP}]
	{snp}{SNP}{SNP - polimorfismo a singolo nucleotide}

\newglossaryentry{pcc}{
	text={coefficiente di correlazione di Pearson},
	name={Coefficiente di correlazione di Pearson}, 
	description={Indice che esprime un'eventuale relazione di linearità tra due insiemi di dati, che consiste nel rapporto tra la covarianza di due variabili e il prodotto delle loro deviazioni standard},
	plural={coefficienti di correlazione di Pearson}
}

\newglossaryentry{esone}{
	text={esone},
	name={Esone}, 
	description={Sequenza di DNA che viene copiata in filamenti di RNA messaggero, e che potrebbe codificare degli amminoacidi. Nel DNA gli esoni sono separati da introni, che invece non vengono copiati},
	plural={esoni}
}

\newglossaryentry{ecotipo}{
	text={ecotipo},
	name={Ecotipo}, 
	description={Popolazione di organismi di una specie che si adatta geneticamente alle caratteristiche dell'habitat in cui vive},
	plural={ecotipi}
}

\newglossaryentry{trasposone}{
	text={trasposone},
	name={Trasposone}, 
	description={Sequenze di DNA in grado di spostarsi da una posizione a un'altra del genoma. Il loro spostamento in genere può creare mutazioni, alterare l'identità genetica di una cellula o modificare la dimensione del genoma},
	plural={trasposoni}
}

\newglossaryentry{clade}{
	text={clade},
	name={Clade}, 
	description={Raggruppamento tassonomico costituito solo da un antenato comune e da tutti i suoi discendenti},
	plural={cladi}
}
