\documentclass[a4paper, 12pt, final, openright, titlepage, twoside]{book}

%permette di aggiungere altri pacchetti dai file inclusi
\usepackage[subpreambles]{standalone} 


%impostazioni lingua
\usepackage[T1]{fontenc}
\usepackage[utf8]{inputenc}
\usepackage[english,italian]{babel}


\input{./resources/pseudocode.tex}

\usepackage[dvipsnames]{xcolor}
\usepackage{graphicx}

\usepackage[italian]{varioref}

%sistema i margini
\usepackage{geometry}
\geometry{a4paper, top=2cm, bottom=2.1cm, inner=3cm, outer=2cm, heightrounded, foot=1.2cm}

%interlinea 1.5
\usepackage{setspace}
\onehalfspacing

%gestione delle testatine
\usepackage{fancyhdr}
\pagestyle{fancy}
\lhead{} \chead{} \rfoot{}
\rhead{} \lfoot{}
\cfoot{\thepage}
\renewcommand{\headrulewidth}{0pt}

%formattazione titoli paragrafo
\usepackage{titlesec}
\titleformat{\chapter}[display]{\normalfont\huge\bfseries}{Capitolo \thechapter}{0.7em}{\huge}

\usepackage{caption}
\captionsetup{format=hang,labelfont={sf,bf}, font=small}

\usepackage{amsmath}
\usepackage[output-decimal-marker={,}]{siunitx}

\usepackage{afterpage}

\usepackage[autostyle,italian=guillemets]{csquotes}
%sorting=nty per citazione in ordine di cognome, backref=true per mostrare dove una fonte è citata
\usepackage[sorting=none, style=numeric, citestyle=numeric-comp, backend=biber]{biblatex}

\addbibresource{bibliografia.bib}

%collegamenti ipertestuali
\usepackage[linktoc=all, colorlinks=true, allcolors=RoyalBlue]{hyperref}

%include il glossario

\usepackage[toc]{glossaries}

\makenoidxglossaries

\setacronymstyle{long-short-desc}

\newglossaryentry{snd}{
	text={distribuzione normale asimmetrica},
	name={Distribuzione normale asimmetrica}, 
	description={Distribuzione continua di probabilità che generalizza la distribuzione normale, permettendo un indice di asimmetria diverso da zero}
}

\newglossaryentry{dbn}{
	text={distribuzione binomiale negativa},
	name={Distribuzione binomiale negativa}, 
	description={Distribuzione discreta di probabilità di parametri $r$ e $p$, che descrive il numero di fallimenti in una sequenza di esperimenti indipendenti, ciascuno con probabilità di successo $p$, prima dell'occorrenza di un numero $r$ di successi},
	plural={distribuzioni binomiali negative}
}

\newglossaryentry{dp}{
	text={distribuzione di Poisson},
	name={Distribuzione di Poisson}, 
	description={Distribuzione discreta di probabilità che esprime la probabilità di un numero dato di eventi che si verificano successivamente e in modo indipendente, fissato un numero medio di occorrenze},
	plural={distribuzioni di Poisson}
}

\newglossaryentry{db}{
	text={distribuzione binomiale},
	name={Distribuzione binomiale}, 
	description={Distribuzione discreta di probabilità di parametri $n$ e $p$, che descrive il numero di successi in una sequenza di $n$ esperimenti indipendenti, ciascuno con probabilità di successo $p$},
	plural={distribuzioni binomiali}
}

\newglossaryentry{locus}{
	text={locus},
	name={Locus}, 
	description={Posizione specifica e fissata all'interno di un cromosoma in cui è posizionato un particolare gene o un'altra sequenza significativa},
	plural={loci}
}

\newglossaryentry{rate_eterozigosity}{
	text={rapporto di eterozigosi},
	name={Rapporto di eterozigosi}, 
	description={Rapporto tra il numero di siti eterozigoti e quello di siti omozigoti nel genoma di un individuo},
	plural={rapporti di eterozigosi}
}

\newglossaryentry{fitting}{
	text={fitting},
	name={Fitting}, 
	description={Costruzione di una curva o di una funzione matematica che abbia la migliore corrispondenza ad una serie di punti. Essa può implicare l'interpolazione, nel caso si richieda un'esatta corrispondenza tra i punti e la funzione da ricercare},
	plural={fitting}
}

\newacronym
	[description={Variazione della sequenza di DNA a carico di un singolo nucleotide, presente almeno nell'1\% degli individui di una popolazione}, longplural={polimorfismi a singolo nucleotide}, name={SNP - Polimorfismo a singolo nucleotide}, shortplural={SNP}]
	{snp}{SNP}{SNP - polimorfismo a singolo nucleotide}

\newglossaryentry{pcc}{
	text={coefficiente di correlazione di Pearson},
	name={Coefficiente di correlazione di Pearson}, 
	description={Indice che esprime un'eventuale relazione di linearità tra due insiemi di dati, che consiste nel rapporto tra la covarianza di due variabili e il prodotto delle loro deviazioni standard},
	plural={coefficienti di correlazione di Pearson}
}

\newglossaryentry{esone}{
	text={esone},
	name={Esone}, 
	description={Sequenza di DNA che viene copiata in filamenti di RNA messaggero, e che potrebbe codificare degli amminoacidi. Nel DNA gli esoni sono separati da introni, che invece non vengono copiati},
	plural={esoni}
}

\newglossaryentry{ecotipo}{
	text={ecotipo},
	name={Ecotipo}, 
	description={Popolazione di organismi di una specie che si adatta geneticamente alle caratteristiche dell'habitat in cui vive},
	plural={ecotipi}
}

\newglossaryentry{trasposone}{
	text={trasposone},
	name={Trasposone}, 
	description={Sequenze di DNA in grado di spostarsi da una posizione a un'altra del genoma. Il loro spostamento in genere può creare mutazioni, alterare l'identità genetica di una cellula o modificare la dimensione del genoma},
	plural={trasposoni}
}

\newglossaryentry{clade}{
	text={clade},
	name={Clade}, 
	description={Raggruppamento tassonomico costituito solo da un antenato comune e da tutti i suoi discendenti},
	plural={cladi}
}


%genera testo casuale
\usepackage{lipsum}

\begin{document}
	
	
	\frontmatter
	\documentclass[crop=false]{standalone}

\usepackage[swapnames]{frontespizio}

\begin{document}
	\begin{frontespizio}
		\Universita{Padova}
		\Logo[3cm]{./resources/images/loghi.jpg}
		\Dipartimento{Ingegneria dell'Informazione}
		\Corso[Laurea]{Ingegneria Informatica}
		\Titolo{\vspace{3cm}Stima della dimensione del genoma tramite k-mers:\\ confronto tra metodi computazionali.}
		\NCandidato{Laureando}
		\Candidato{Mattia Tamiazzo}
		\Relatore{Prof. Matteo Comin}
		\Rientro{1cm}
		\Annoaccademico{2021-2022}
		\Preambolo{\renewcommand{\frontlogosep}{15pt}}
		\Margini{1.5cm}{1cm}{1.5cm}{1cm}
		\Preambolo{\renewcommand{\frontnamesfont}{%
				\fontsize{12}{14}\bfseries\vspace{2cm}}}
	\end{frontespizio}
	\cleardoublepage
\end{document}

	\documentclass[crop=false]{standalone}


\begin{document}
	
	\chapter*{\abstractname}
		La dimensione del genoma è la quantità totale di DNA nucleare aploide presente nelle cellule di un organismo. La determinazione della dimensione del genoma costituisce un argomento di interesse, perché non esistono valori di riferimento assoluti che permettano di stabilire quale approccio sia più efficace, e perché i metodi sperimentali per la sua misurazione sono attualmente costosi dal punto di vista temporale ed economico. Una soluzione alla stima della dimensione del genoma con metodi computazionali è l'utilizzo di k-mers, sottostringhe di DNA di lunghezza k. Questa trattazione si pone l'obiettivo di analizzare e comparare vari approcci algoritmici pubblicati in letteratura per la stima della dimensione del genoma, che tuttavia spesso forniscono risultati contrastanti e di difficile valutazione assoluta. Grazie ai limitati requisiti richiesti rispetto ai metodi attuali, comunque, la stima basata sui k-mer risulta essere un approccio promettente, anche per lo sviluppo futuro di metodi più precisi ed efficienti.
		
	\cleardoublepage
\end{document}

	\documentclass[crop=false]{standalone}

\setcounter{tocdepth}{1}
\usepackage{hyperref}

\begin{document}
	\hypersetup{linkcolor=black}
	\tableofcontents
\end{document}
	
	
	
	\mainmatter
	\documentclass[crop=false, class=book]{standalone}

\usepackage{lipsum}

\begin{document}
	\chapter{Introduzione}
	
	Il sequenziamento del DNA costituisce una tecnica fondamentale per lo studio del genoma di una specie, perché permette di determinare l'ordine delle basi azotate dei nucleotidi che costituiscono il DNA. Tale processo trova applicazione in molti studi biologici che riguardano vari ambiti, come ad esempio la medicina riproduttiva, l'oncologia o l'infettivologia, attraverso indagini tra cellule diverse dello stesso individuo o lo studio delle mutazioni genetiche tra individui di una stessa specie \cite{shendure2012expanding}. 
	
	% CONTENUTI (ordine da determinare)
	% a cosa serve (es: Assembly)
	% breve storia 
	% metodi (sperimentali vs computazionali) -> k-mer
	% cosa sono i k-mer
	
	\section{Storia}
	Lo studio approfondito del DNA si sviluppa a partire dal 1953, con la scoperta della sua struttura tridimensionale ad opera di James Watson e Francis Crick \cite{watson1953molecular}, contribuendo all'analisi dell'azione degli acidi nucleici nella sintesi proteica. Solo nel 1977 però, vennero sviluppate le prime strategie sperimentali per il sequenziamento, come il famoso metodo Sanger \cite{sanger1977DNA, sanger1977nucleotide} TODO
	
	
	\section{Sequenze di lunghezza k: i k-mer}
		TODO
		
	\subsection{K-mer profile}
	\label{subsec:kmerprofile}
	Il \textit{k-mer profile}, detto anche \textit{k-mer spectrum}, conta la frequenza dei k-mer trovati nelle letture di input, non assemblate o allineate. Esso può rappresentare un indicatore della complessità del genoma preso in esame \cite{vurture2017genomescope}, e mostra la quantità di k-mer distinti trovati ad una certa frequenza. Un esempio di k-mer profile è mostrato dalla figura~\vref{fig:profilecomp} tratta da \cite{sohn2016present}, in cui si può notare come la natura del genoma influenzi direttamente il grafico. 

	
	\begin{figure}
		\centering
		\includegraphics[width=0.8\textwidth]{capitoli/introduzione/profilecomp.png}
		\caption{TODO.}
		\label{fig:profilecomp}
	\end{figure}
	
	Ipotizzando che il genoma sia ideale, omozigote e senza ripetizioni, e che le letture siano state fatte senza errori con una certa copertura, il grafico del k-mer profile sarà una \gls{dp} centrata sulla copertura media disponibile.
	
	In casi reali invece, il genoma sarà eterozigote con una certa percentuale di eterozigosi e saranno presenti errori di sequenziamento; il k-mer profile presenterà tre picchi principali~\cite{sun2017findGSE}.
	Il primo picco del grafico corrisponde ai k-mer derivati da errori di sequenziamento, che accadono spesso ma che hanno bassa frequenza perché presentano poche occorrenze nelle letture di input; il secondo invece, rappresenta i k-mer eterozigoti e il terzo quelli omozigoti, presenti quindi su uno o entrambi gli alleli del set di cromosomi. I k-mer eterozigoti devono essere trattati più attentamente, perché possono risultare simili a quelli del primo picco, derivanti da errori di sequenziamento~\cite{sohn2016present}.	
	
	La lunga coda della distribuzione rappresenta invece le sequenze ripetitive, che occorrono con alta frequenza e sono presenti in un elevato numero di \gls{locus}. Eventuali ripetizioni aggiungono al grafico ulteriori picchi, mentre errori nelle letture aumentano la varianza e producono distorsioni nel grafico.
	
	La figura~\vref{fig:kmerprofile} mostra come all'aumentare del \gls{rate_eterozigosity} la quantità di k-mer eterozigoti del secondo picco diventi dominante rispetto ai k-mer omozigoti del terzo picco, che invece diminuiscono.
	
	Il k-mer profile può essere calcolato tramite programmi specifici date delle letture del genoma di input, quali \textit{Jellyfish}~\cite{marcais2011fast} o \textit{KMC2}~\cite{deorowicz2015KMC}.
	
	
	
	
	\begin{figure}
		\centering
		\includegraphics[width=0.8\textwidth]{capitoli/introduzione/kmerprofile.png}
		\caption{TODO.}
		\label{fig:kmerprofile}
	\end{figure}

	%\lipsum[1]
	%\section{Lorem ipsum}
	%\lipsum[2]
	%\subsection{Dolor sit amet} 
	%\lipsum[3]
	%\subsubsection{Lorem ipsum}
	%\lipsum[3]
	
	
	
	
\end{document}
	\chapter{Metodi analizzati}
	In questo capitolo vengono presentati, in ordine cronologico di pubblicazione, i metodi presi in esame. Ciascun programma viene descritto svolgendo un'analisi dell'algoritmo che lo caratterizza, ed elencando le funzionalità previste per la gestione di eventi particolari, come ad esempio la presenza di errori di sequenziamento. 
	\documentclass[crop=false, class=book]{standalone}

\begin{document}
	
	
	\section{ALLPATHS-LG}
	\label{sec:allpaths}
	\textit{ALLPATHS-LG} è un programma che permette di eseguire il sequenziamento di un genoma tramite de novo assembly di letture shotgun, e che calcola implicitamente la dimensione totale del genoma. Esso si basa sul programma \textit{ALLPATHS}~\cite{butler2008allpaths,maccallum2009allpaths2} sviluppato precedentemente e, rispetto al suo predecessore, permette l'assembly di genomi di dimensioni maggiori e con copertura minore, di gestire sequenze ripetitive, di correggere errori di lettura e di utilizzare in modo più efficiente le risorse disponibili durante il sequenziamento~\cite{gnerre2011high}. 
	
	
	\subsection{Algoritmo}
	L'algoritmo del programma si basa sul precedente software ALLPATHS~\cite{butler2008allpaths}. Dato un numero minimo $k$ di basi che si sovrappongono nelle letture shotgun, si definisce \textit{branch} una sequenza di $k$ basi (k-mer) che compare in due o più letture diverse, e la cui base successiva o precedente è diversa in ogni lettura. Spezzando il genoma in corrispondenza di ciascun branch, esso viene scomposto in un insieme di sequenze, dette \textit{unipath}. Tali sequenze vengono create a partire dalle letture shotgun di input non allineate. 
	
	\paragraph{Formazione dei k-mer path}
	Inizialmente negli shotgun viene corretto il maggior numero di errori di lettura, e vengono poi riconosciuti tutti i k-mer di lunghezza $k$. In ogni sequenza, ciascun k-mer viene numerato con un numero intero unico; a k-mer già trovati viene assegnato lo stesso valore. In questo modo ciascuna lettura potrà essere espressa come una sequenza di numeri interi, ognuno dei quali rappresenta un k-mer. 
	
	I numeri della sequenza vengono poi raggruppati in intervalli, in modo da formare i \textit{k-mer path}; essi associano ciascun intervallo al k-mer path a cui esso appartiene, permettendo una facile ricerca di tutte le sequenze che condividono un certo k-mer e semplificando quindi l'assembly delle letture shotgun. Si veda la figura~\vref{fig:allpathsnumbering} per un esempio di come avviene la numerazione dei k-mer e la creazione dei k-mer path.
	Le letture così tradotte costituiscono il database in cui è possibile effettuare ricerche per la costruzione degli unipath.
	
	\begin{figure}
		\centering
		\includegraphics[width=0.6\textwidth]{capitoli/metodi analizzati/allpaths/numbering.png}
		\caption{Esempio di numerazione dei k-mer e traduzione della sequenza in intervalli. Posto $k=6$, vengono individuati tutti i k-mer presenti, ognuno dei quali viene numerato con un numero unico, riutilizzandolo nel caso di k-mer ripetuti. Usando gli intervalli, la sequenza iniziale viene quindi tradotta in $([100,112], [101, 102], [113, 114])$.}
		\label{fig:allpathsnumbering}
	\end{figure}
	
	\paragraph{Formazione degli unipath}
	Tutti i numeri degli intervalli trovati nei k-mer path vengono scanditi. L'obiettivo per ogni numero è trovare il più lungo intervallo senza interruzioni che lo contenga. Si cercano nel database tutti gli intervalli che contengono quel numero, e si sceglie l'intervallo continuo più lungo, che diventa un \textit{unipath interval}. Il processo è ripetuto per tutti i k-mer che non sono stati ancora inclusi in un unipath interval. 
	
	Per ogni unipath interval viene preso il primo numero di k-mer; esso viene cercato nel database e, a partire da quest'ultimo, si determina, se presente, un suo possibile predecessore. Se ne esiste esattamente uno, l'unipath interval del primo numero viene collegato all'intervallo alla sua sinistra. Procedendo iterativamente, viene formato un unipath, che consiste quindi in un gruppo di k-mer contigui. 
	
	\paragraph{Isolamento dei seed unipath}
	Tra gli unipath creati, è possibile isolare dei \textit{seed unipath} attorno ai quali poi eseguire l'assembly. Preso l'insieme di tutti gli unipath, iterativamente si rimuovono alcuni di essi; quindi, preso un certo unipath $u$, si isolano quelli adiacenti a destra e a sinistra rispetto a $u$. Tra i due vicini di $u$ viene dedotta la distanza; se essa è minore di una certa soglia, $u$ può essere rimosso dall'insieme. Si procede in questo modo per tutti gli unipath presenti finché continua a essere possibile la rimozione. Gli unipath rimasti costituiranno un seed unipath.
	
	\paragraph{Assembly locale e globale}
	Attorno ai seed unipath viene quindi assemblato il vicinato (\textit{neighborhood}), cioè le regioni di 10 kb che precedono e seguono il seme. Per farlo, vengono creati due gruppi di sequenze, il primo (\textit{primary read cloud}) contenente letture la cui posizione reale è vicina al seme, mentre il secondo (\textit{secondary read cloud}) contiene brevi frammenti la cui sequenza può essere assemblata con le letture del primo gruppo. Il vicinato dei semi viene quindi assemblato utilizzando i due gruppi di letture, formando un \textit{sequence graph} dell'assembly locale, cioè attorno al seed unipath corrispondente.
	
	I vari assembly locali vengono fatti in parallelo e sono poi uniti per formare un sequence graph unico, relativo cioè all'assembly globale.
	
	
	\subsection{Gestione di genomi di grandi dimensioni}
	I predecessori di ALLPATHS-LG ottengono risultati promettenti solo per genomi di piccole dimensioni. Per gestire genomi di dimensioni più grandi, come quelli dei mammiferi, sono state fatte delle modifiche notevoli~\cite{gnerre2011high}.

	Il programma cerca di comprimere le ripetizioni in modo da favorirne l'allineamento. Se una sequenza ripetitiva è presente in due letture separate, il programma utilizza un'altra coppia di shotgun per poter allineare le prime due (\textit{gap filling}). Le letture vengono unite se un'altra coppia fornisce una sovrapposizione, e viceversa. Il metodo può essere utilizzato anche se sono presenti mutazioni \gls{snp}, che forniscono più soluzioni di assemblamento. La figura~\vref{fig:allpathsfilling} tratta da~\cite{gnerre2011high}, mostra come avviene la gestione delle sequenze ripetitive nei due diversi casi.

	\begin{figure}
		\centering
		\subfloat[][\emph{L'allineamento della coppia di letture in nero è possibile usando un'altra coppia di letture, raffigurate in rosso.}]
		{\includegraphics[width=0.6\textwidth]{capitoli/metodi analizzati/allpaths/filling1.png}} \\
		\subfloat[][\emph{Le due coppie di letture in rosso potrebbero allineare la coppia di letture in nero, ma presentano un SNP. Vengono quindi mantenute entrambe, fornendo due diverse soluzioni possibili.}]
		{\includegraphics[width=0.6\textwidth]{capitoli/metodi analizzati/allpaths/filling2.png}} 
		\caption{Esempi grafici del processo di gap filling.}
		\label{fig:allpathsfilling}
	\end{figure}
	
	La nuova versione del programma, inoltre, migliora la correzione degli errori di lettura: dato un k-mer, vengono analizzate tutte le letture che lo contengono; nel caso in cui una singola lettura differisca dalla maggior parte delle altre, essa viene corretta se non ci sono voti in conflitto, altrimenti non viene modificata.
	
	Ulteriori miglioramenti sono stati applicati anche per la gestione di sequenze a bassa copertura: dato che in questi casi la sovrapposizione tra letture può essere breve, in tali zone è preferibile utilizzare k-mer di dimensioni minori cioè scegliere un valore di $k$ più basso. Il programma permette di utilizzare un valore di $k>15$, ma solo nelle regioni che verrebbero assegnate a uno spazio vuoto tra altre due sequenze. 
	
	\subsection{Stima della dimensione del genoma}
	Il programma, pur avendo come scopo primario la costruzione dell'assembly a partire dalle letture del genoma, produce in output un valore stimato della sua dimensione. Per farlo, dato il grafico del k-mer profile, esso identifica l'ascissa del punto di flesso $f_v$ che si trova tra il primo e il secondo picco, cioè tra quello dei k-mer derivanti da errori di sequenziamento e quello dei k-mer omozigoti, e l'ascissa $f_p$ corrispondente al picco eterozigote \cite{sun2017findGSE}.
	
	La dimensione del genoma $G$ viene quindi stimata considerando gli $N$ k-mer con copertura $C$ compresi tra $f_v$ e $3f_p/2$, scartando quindi dalle letture i k-mer a frequenze basse o molto alte, tramite la formula $G = N/C$.

\end{document}
	\documentclass[crop=false, class=book]{standalone}

\usepackage{lipsum}

\begin{document}
	\chapter{GCE}
	
%	Il programma \textit{findGSE} \cite{sun2017findGSE} ha come obiettivo principale la stima della lunghezza del genoma. Utilizzando le frequenze dei k-mer trovati nelle letture a disposizione, il programma compie una regressione non lineare dei dati utilizzando come funzione una \gls{snd} (\textit{skew normal distribution} \cite{azzalini1985class,azzalini2005skew}).
%	
%	
%	\section{Algoritmo}
%	Nel programma viene assunto che le frequenze dei k-mer possano essere approssimate da una distribuzione normale asimmetrica $SN(\xi, \omega^2, \alpha)$. Presa in input la distribuzione delle frequenze dei k-mer (k-mer profile), l'algoritmo effettua la regressione determinando i quattro parametri che descrivono una distribuzione normale asimmetrica, la media $\xi$, la deviazione standard $\omega$, l'asimmetria $\alpha$ e un fattore di scala $s$. Ad ogni iterazione, il programma cerca di minimizzare l'errore tra i dati di input e la funzione stimata, in modo da approssimare il più possibile il k-mer profile reale. 
%	
%	Dato un genoma aploide con $G$ basi, il numero di k-mer possibili sarà $G-k+1$. Ponendo $C$ la copertura media dei k-mer, cioè che in media ogni k-mer sia trovato in $C$ letture diverse, e $N$ il numero di k-mer trovati nelle letture, la quantità di k-mer presenti nel genoma è descritta dall'equazione~\vref{eqn:findGSE1}. 	
%	
%	\begin{equation}
%		\label{eqn:findGSE1}
%		N=C*(G-K+1)
%	\end{equation}
%		
%	
%	Posta la dimensione del genoma molto maggiore del numero di basi utilizzate $G\gg k$, l'equazione~\vref{eqn:findGSE2} approssima la dimensione totale del genoma in analisi.
%	
%	\begin{equation}
%		\label{eqn:findGSE2}
%		G\approx N/C
%	\end{equation}
%
%	A partire sia dal profilo reale che dal modello stimato, il programma calcola quindi il numero totale di k-mer trovati $N$ e la copertura media dei k-mer $C$, per poi calcolare la dimensione del genoma attraverso l'equazione~\vref{eqn:findGSE2}.
%	

	%TODO risultati?
	

	
	
	
\end{document}
	\documentclass[crop=false, class=book]{standalone}

\usepackage{lipsum}

\begin{document}
	\section{GenomeScope}
	
	Il progetto open source \textit{GenomeScope} cerca sia di stimare le caratteristiche del genoma completo, come ad esempio la sua lunghezza o il rapporto di eterozigosi, sia di determinare le proprietà delle letture di DNA che prende in input, come la copertura (\textit{read coverage}) o l'error rate~\cite{vurture2017genomescope}. Il programma per determinare tali caratteristiche utilizza il k-mer profile del genoma preso in esame, descritto nella sezione~\vref{subsec:kmerprofile}.
		

	\subsection{Algoritmo}
	Il programma effettua una regressione non lineare dei dati iniziali, generando un profilo che cerca di approssimare il k-mer profile reale. Prendendo in input le letture del genoma che si vuole studiare, esso crea un modello che approssima il più possibile il k-mer profile. La funzione $f(X)$ scelta per l'interpolazione delle frequenze dei k-mer trovati è la somma di quattro \glspl{dbn} $Y \sim \mathcal{NB}(X;p,n)$, per rappresentare rispettivamente i k-mer eterozigoti trovati nel genoma diploide una volta (unici) o tre volte (duplicati), e i k-mer omozigoti di cui si trovano due occorrenze (unici) o trovati quattro volte (duplicati). La funzione $f(X)$ è descritta dall'equazione~\vref{eqn:gnmscp_regression}, in cui $G$ rappresenta un coefficiente di scala legato alla dimensione del genoma, $\lambda$ e $\rho$ sono rispettivamente la media e la varianza della distribuzione. 
	\begin{multline}
		f(X) = G \times (\alpha \mathcal{NB}(X;\lambda, \lambda/\rho) + \beta \mathcal{NB}(X;2\lambda, 2\lambda/\rho) + \\
		\gamma \mathcal{NB}(X;3\lambda, 3\lambda/\rho) + \delta \mathcal{NB}(X;4\lambda, 4\lambda/\rho)  ).	
		\label{eqn:gnmscp_regression}
	\end{multline}

	I coefficienti $\alpha, \beta, \gamma$ e $\delta$ dipendono dai parametri $r$ e $d$, che rappresentano rispettivamente il rapporto di eterozigosi, cioè la percentuale di basi che sono specifiche a uno o due cromosomi omologhi, e la percentuale del genoma che è presente in due copie.
	
	Lo scopo del programma è quindi determinare i coefficienti $r, d, \lambda$ e $\rho$, oltre alla dimensione totale del genoma $G$. La funzione scelta $f(X)$, tramite cui poi può essere calcolata la dimensione del genoma, è quella che restituisce la minore somma dei quadrati degli errori residui (\textit{Residual Sum of Square Error} - \textit{RSSE}), che cioè minimizzi la somma tra i quadrati degli errori tra i valori osservati e quelli stimati, come descritto dall'equazione~\vref{eqn:gnmscp_RSSE}. Per dedurre i valori dei coefficienti, viene utilizzata la funzione \verb|nls| del linguaggio di programmazione \textit{R}, che compie il \gls{fitting} dei dati alla funzione obiettivo.
	\begin{equation}
		RSSE = \sum_{x=E}^{+\infty} \left(kmer_{obs}[x] - kmer_{pred}[x]\right)^2.
	\label{eqn:gnmscp_RSSE}
	\end{equation}
	Al termine, il programma mostra all'utente i dati relativi al genoma trovati, come il rapporto di eterozigosi, la media e la varianza della distribuzione, l'indice RSSE, che rappresenta la percentuale di k-mer non considerati dal modello, e la dimensione stimata del genoma.
	
	La figura~\vref{fig:gnmscp_genomescopeprofile} mostra un confronto tra il k-mer profile reale e il modello costruito dal programma.
	
	\begin{figure}
		\centering
		\includegraphics[width=0.6\textwidth]{capitoli/genomescope/gnmscp_genomescopeprofile.png}
		\caption{Modello del k-mer profile generato dal programma. L'istogramma in azzurro rappresenta i dati del k-mer profile reale, la curva nera il modello stimato e quella color arancione gli errori di sequenziamento stimati.}
		\label{fig:gnmscp_genomescopeprofile}
	\end{figure}


	\subsection{Gestione degli errori di sequenziamento}
	Eventuali errori di sequenziamento, ad esempio dovuti a duplicazioni con PCR o a sequenze contaminate, sono determinati solo empiricamente: dopo varie iterazioni del software in cui viene abbassata la soglia di copertura richiesta, i k-mer che non riescono ad essere rappresentati dal modello vengono identificati come errori di sequenziamento. 
	
	La stima dei k-mer sequenziati in modo errato è importante perché viene utilizzata per determinare la percentuale di basi errate nelle letture: una singola base inesatta infatti può dar luogo fino a $k$ k-mer errati, aumentando notevolmente il numero di errori. GenomeScope permette una percentuale $e$ di basi errate in ciascun k-mer. Tale valore è calcolato con un fitting dei k-mer errati a una distribuzione binomiale tramite la funzione \verb|uniroot| del linguaggio di programmazione \textit{R}. Questo metodo permette al programma di non dover assumere che la distribuzione degli errori di sequenziamento abbia una particolare forma, né di utilizzare un valore di soglia.
	
\end{document}
	\documentclass[crop=false, class=book]{standalone}

\usepackage{lipsum}

\begin{document}
	\chapter{findGSE}
	
	Il programma \textit{findGSE} \cite{sun2017findGSE} ha come obiettivo principale la stima della lunghezza del genoma. Utilizzando le frequenze dei k-mer trovati nelle letture a disposizione, il programma compie una regressione non lineare dei dati utilizzando come funzione una \gls{snd} (\textit{skew normal distribution} \cite{azzalini1985class,azzalini2005skew}).
	
	
	\section{Algoritmo}
	Dato un genoma aploide con $G$ basi, il numero di k-mer possibili sarà $G-k+1$. Ponendo $C$ la copertura dei k-mer, cioè che in media ogni k-mer sia trovato in $C$ letture diverse, e $N$ il numero di k-mer trovati nelle letture, la quantità di k-mer presenti nel genoma sarà $N=C*(G-K+1)$. Dall'equazione si deduce che $G\approx N/C$ se $G\gg k$.
	
	Nel programma viene assunto che le frequenze dei k-mer possano essere approssimate da una distribuzione normale asimmetrica $SN(\xi, \omega^2, \alpha)$. Presa in input la distribuzione delle frequenze dei k-mer (k-mer profile), l'algoritmo effettua la regressione determinando i quattro parametri che descrivono una distribuzione normale asimmetrica, la media $\xi$, la deviazione standard $\omega$, l'asimmetria $\alpha$ e un fattore di scala $s$. Ad ogni iterazione, il programma cerca di minimizzare l'errore tra i dati di input e la funzione stimata, in modo da approssimare il più possibile il k-mer profile reale.
	

%	Il programma effettua una regressione non lineare dei dati del k-mer profile, generando un nuovo profilo che cerca di approssimare il k-mer profile reale. Prendendo in input le letture del genoma che si vuole studiare, esso crea un modello che approssimi il più possibile il k-mer profile. La funzione $f(X)$ scelta per l'interpolazione delle frequenze dei k-mer trovati è la somma di quattro distribuzioni binomiali negative $\mathcal{NB}(X;p,n)$, rispettivamente per rappresentare k-mer eterozigoti trovati nel genoma diploide una volta (unici) o tre volte (duplicati), e k-mer omozigoti di cui si trovano due occorrenze (unici) o trovati quattro volte (duplicati). La funzione $f(X)$ è descritta dall'equazione~\vref{eqn:gnmscp_regression}, in cui $G$ rappresenta un coefficiente di scala legato alla dimensione del genoma, $\lambda$ e $\rho$ sono rispettivamente la media e la varianza della distribuzione. 
%	\begin{multline}
%		f(X) = G * (\alpha \mathcal{NB}(X;\lambda, \lambda/\rho) + \beta \mathcal{NB}(X;2\lambda, 2\lambda/\rho) + \\
%		\gamma \mathcal{NB}(X;3\lambda, 3\lambda/\rho) + \delta \mathcal{NB}(X;4\lambda, 4\lambda/\rho)  )	
%		\label{eqn:gnmscp_regression}
%	\end{multline}
%
%	I coefficienti $\alpha, \beta, \gamma$ e $\delta$ dipendono dai parametri $r$ e $d$, che rappresentano rispettivamente il rapporto di eterozigosi, cioè la percentuale di basi che sono specifiche a uno o due cromosomi omologhi, e la percentuale del genoma che è presente in due copie.
%	
%	Lo scopo del programma è quindi determinare i coefficienti $r, d, \lambda$ e $\rho$, oltre alla dimensione totale del genoma $G$. La funzione scelta $f(X)$, tramite cui poi può essere calcolata la dimensione del genoma, è quella che restituisce la minore somma dei quadrati degli errori residui (\textit{Residual Sum of Square Error} - \textit{RSSE}), cioè che minimizzi la somma tra i quadrati degli errori tra i valori osservati e quelli stimati, come descritto dall'equazione~\vref{eqn:gnmscp_RSSE}. Per dedurre i valori dei coefficienti, viene utilizzata la funzione \verb|nls| del linguaggio di programmazione R, che compie la regressione non lineare dei dati alla funzione obiettivo.
%	\begin{equation}
%		RSSE = \sum_{x=E}^{+\infty} \left(kmer_{obs}[x] - kmer_{pred}[x]\right)^2
%	\label{eqn:gnmscp_RSSE}
%	\end{equation}
%	Al termine, il programma mostra all'utente i dati relativi al genoma trovati, come il rapporto di eterozigosi, la media e la varianza della distribuzione, l'indice RSSE, che rappresenta la percentuale di k-mer non considerati dal modello, e la dimensione stimata del genoma.
%	
%	Eventuali errori di sequenziamento, ad esempio dovuti a duplicazioni con PCR o a sequenze contaminate, sono determinati solo empiricamente: dopo varie iterazioni del software in cui viene abbassata la soglia di copertura richiesta, i k-mer che non riescono ad essere rappresentati dal modello vengono identificati come errori di sequenziamento. 
%	
%	\begin{figure}
%		\centering
%		\includegraphics[width=0.6\textwidth]{capitoli/genomescope/gnmscp_genomescopeprofile.png}
%		\caption{TODO + TODO reference a figura.}
%		\label{fig:gnmscp_genomescopeprofile}
%	\end{figure}

	%TODO risultati?
	

	
	
	
\end{document}
	\documentclass[crop=false, class=book]{standalone}

\usepackage{lipsum}

\begin{document}
	\section{MGSE}
	Il programma \textit{Mapping-based Genome Size Estimation} (\textit{MGSE}) stima la dimensione del genoma attraverso l'assembly delle letture utilizzando un genoma di riferimento ad alta contiguità \cite{pucker2019MGSE} TODO. Lo script è open-source e scritto in Python, e processa le informazioni sulla copertura delle letture di input restituendo la dimensione stimata del genoma.
	
	\subsection{Algoritmo}
	Posto che le letture di input siano distribuite equamente sull'intera sequenza del genoma, il programma ne stima la dimensione calcolandone la copertura media. Se infatti sono noti il numero $L$ di basi sequenziate e la copertura $C$ in una certa posizione, la lunghezza totale $N$ del genoma sarà pari a $N = L/C$.
	Dato che le letture sono 
	
	

	
	
	
\end{document}
	
	\documentclass[crop=false, class=book]{standalone}

\usepackage{lipsum}
\usepackage{subfig}

\begin{document}
	
	\chapter{Confronto tra i metodi}
	In questo capitolo TODO
	
	\section{Complessità e performance}
	Vengono analizzati per ciascun metodo i ..., criteri utilizzati per numero di file/LOC (solo codice, esclusi commenti/bianchi) TODO
	
	\subsection{ALLPATHS-LG}
	Dato che il programma ha come obiettivo principale l'assembly delle letture di input, esso presenta una complessità elevata (1026 file, 215400 righe di codice). Inoltre sono necessarie prestazioni elevate (48 processori, 512 GB di RAM) e tempi lunghi di processamento TODO
	
	Non è il metodo migliore perché richiede grande sforzo computazionale, ma può servire da reference TODO.
	
	
		
	
\end{document}
	
	
	
	\backmatter
	\documentclass[crop=false]{standalone}




\begin{document}

	\printnoidxglossary[sort=word]

\end{document}
	
	\input{fine/bibliografia.tex}
	
\end{document}
