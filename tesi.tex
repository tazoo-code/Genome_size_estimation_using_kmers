\documentclass[a4paper, 12pt, final, openany, titlepage, twoside]{book}

%permette di aggiungere altri pacchetti dai file inclusi
\usepackage[subpreambles]{standalone} 


%impostazioni lingua
\usepackage[T1]{fontenc}
\usepackage[utf8]{inputenc}
\usepackage[english,italian]{babel}

\usepackage[italian]{varioref}

%sistema i margini
\usepackage{geometry}
\geometry{a4paper, top=2cm, bottom=2cm, left=3cm, right=2cm, heightrounded}

%interlinea 1.5
\usepackage{setspace}
\onehalfspacing

%gestione delle testatine
\usepackage{fancyhdr}
\pagestyle{fancy}
\lhead{} \chead{} \rfoot{}
\rhead{} \lfoot{}
\cfoot{\thepage}
\renewcommand{\headrulewidth}{0pt}

%formattazione titoli paragrafo
\usepackage{titlesec}
\titleformat{\chapter}[display]{\normalfont\huge\bfseries}{Capitolo \thechapter}{0.7em}{\huge}

\usepackage{caption}

\usepackage{afterpage}


\usepackage[autostyle,italian=guillemets]{csquotes}
\usepackage[sorting=nty, style=numeric, citestyle=numeric-comp, backend=biber]{biblatex}

\addbibresource{bibliografia.bib}

%collegamenti ipertestuali
\usepackage[linktoc=all, colorlinks=true, allcolors=RoyalBlue]{hyperref}

%genera testo casuale
\usepackage{lipsum}

\begin{document}
	
	
	\frontmatter
	\documentclass[crop=false]{standalone}

\usepackage[swapnames]{frontespizio}

\begin{document}
	\begin{frontespizio}
		\Universita{Padova}
		\Logo[3cm]{./resources/images/loghi.jpg}
		\Dipartimento{Ingegneria dell'Informazione}
		\Corso[Laurea]{Ingegneria Informatica}
		\Titolo{\vspace{3cm}Stima della dimensione del genoma tramite k-mers:\\ confronto tra metodi computazionali.}
		\NCandidato{Laureando}
		\Candidato{Mattia Tamiazzo}
		\Relatore{Prof. Matteo Comin}
		\Rientro{1cm}
		\Annoaccademico{2021-2022}
		\Preambolo{\renewcommand{\frontlogosep}{15pt}}
		\Margini{1.5cm}{1cm}{1.5cm}{1cm}
		\Preambolo{\renewcommand{\frontnamesfont}{%
				\fontsize{12}{14}\bfseries\vspace{2cm}}}
	\end{frontespizio}
	\cleardoublepage
\end{document}

	\documentclass[crop=false]{standalone}


\begin{document}
	
	\chapter*{\abstractname}
		La dimensione del genoma è la quantità totale di DNA nucleare aploide presente nelle cellule di un organismo. La determinazione della dimensione del genoma costituisce un argomento di interesse, perché non esistono valori di riferimento assoluti che permettano di stabilire quale approccio sia più efficace, e perché i metodi sperimentali per la sua misurazione sono attualmente costosi dal punto di vista temporale ed economico. Una soluzione alla stima della dimensione del genoma con metodi computazionali è l'utilizzo di k-mers, sottostringhe di DNA di lunghezza k. Questa trattazione si pone l'obiettivo di analizzare e comparare vari approcci algoritmici pubblicati in letteratura per la stima della dimensione del genoma, che tuttavia spesso forniscono risultati contrastanti e di difficile valutazione assoluta. Grazie ai limitati requisiti richiesti rispetto ai metodi attuali, comunque, la stima basata sui k-mer risulta essere un approccio promettente, anche per lo sviluppo futuro di metodi più precisi ed efficienti.
		
	\cleardoublepage
\end{document}

	\documentclass[crop=false]{standalone}

\setcounter{tocdepth}{1}
\usepackage{hyperref}

\begin{document}
	\hypersetup{linkcolor=black}
	\tableofcontents
\end{document}
	
	
	
	\mainmatter
	\documentclass[crop=false, class=book]{standalone}

\usepackage{lipsum}

\begin{document}
	\chapter{Introduzione}
	
	Il sequenziamento del DNA costituisce una tecnica fondamentale per lo studio del genoma di una specie, perché permette di determinare l'ordine delle basi azotate dei nucleotidi che costituiscono il DNA. Tale processo trova applicazione in molti studi biologici che riguardano vari ambiti, come ad esempio la medicina riproduttiva, l'oncologia o l'infettivologia, attraverso indagini tra cellule diverse dello stesso individuo o lo studio delle mutazioni genetiche tra individui di una stessa specie \cite{shendure2012expanding}. 
	
	% CONTENUTI (ordine da determinare)
	% a cosa serve (es: Assembly)
	% breve storia 
	% metodi (sperimentali vs computazionali) -> k-mer
	% cosa sono i k-mer
	% spiegare eterozigosi/omozigosi
	
	\section{Storia}
	Lo studio approfondito del DNA si sviluppa a partire dal 1953, con la scoperta della sua struttura tridimensionale ad opera di James Watson e Francis Crick \cite{watson1953molecular}, contribuendo all'analisi dell'azione degli acidi nucleici nella sintesi proteica. Solo nel 1977 però, vennero sviluppate le prime strategie sperimentali per il sequenziamento, come il famoso metodo Sanger \cite{sanger1977DNA, sanger1977nucleotide}
	
	%\lipsum[1]
	%\section{Lorem ipsum}
	%\lipsum[2]
	%\subsection{Dolor sit amet} 
	%\lipsum[3]
	%\subsubsection{Lorem ipsum}
	%\lipsum[3]
	
	% GLOSSARIO
	% omozigosi/eterozigosi
	% distribuzione binomiale negativa
	
	
	
	
\end{document}
	
	
	
	\backmatter
	\input{fine/bibliografia.tex}
	
\end{document}
