\documentclass[crop=false]{standalone}


\begin{document}
	
	\chapter*{\abstractname}
		La dimensione del genoma è la quantità totale di DNA nucleare aploide presente nelle cellule di un organismo. La determinazione della dimensione del genoma costituisce un argomento di interesse, perché non esistono valori di riferimento assoluti che permettano di stabilire quale approccio sia più efficace, e perché i metodi sperimentali per la sua misurazione sono attualmente costosi dal punto di vista temporale ed economico. Una soluzione alla stima della dimensione del genoma con metodi computazionali è l'utilizzo di k-mers, sottostringhe di DNA di lunghezza k. Questa trattazione si pone l'obiettivo di analizzare e comparare vari approcci algoritmici pubblicati in letteratura per la stima della dimensione del genoma, che tuttavia spesso forniscono risultati contrastanti e di difficile valutazione assoluta. Grazie ai limitati requisiti richiesti rispetto ai metodi attuali, comunque, la stima basata sui k-mer risulta essere un approccio promettente, anche per lo sviluppo futuro di metodi più precisi ed efficienti.
		
	\cleardoublepage
\end{document}
