\documentclass[crop=false, class=book]{standalone}

\usepackage{lipsum}

\begin{document}
	\chapter{findGSE}
	
	Il programma \textit{findGSE} \cite{sun2017findGSE} ha come obiettivo principale la stima della lunghezza del genoma. Utilizzando le frequenze dei k-mer trovati nelle letture a disposizione, il programma compie una regressione non lineare dei dati utilizzando come funzione una \gls{snd} (\textit{skew normal distribution} \cite{azzalini1985class,azzalini2005skew}).
	
	
	\section{Algoritmo}
	Dato un genoma aploide con $G$ basi, il numero di k-mer possibili sarà $G-k+1$. Ponendo $C$ la copertura dei k-mer, cioè che in media ogni k-mer sia trovato in $C$ letture diverse, e $N$ il numero di k-mer trovati nelle letture, la quantità di k-mer presenti nel genoma sarà $N=C*(G-K+1)$. Dall'equazione si deduce che $G\approx N/C$ se $G\gg k$.
	
	Nel programma viene assunto che le frequenze dei k-mer possano essere approssimate da una distribuzione normale asimmetrica $SN(\xi, \omega^2, \alpha)$. Presa in input la distribuzione delle frequenze dei k-mer (k-mer profile), l'algoritmo effettua la regressione determinando i quattro parametri che descrivono una distribuzione normale asimmetrica, la media $\xi$, la deviazione standard $\omega$, l'asimmetria $\alpha$ e un fattore di scala $s$. Ad ogni iterazione, il programma cerca di minimizzare l'errore tra i dati di input e la funzione stimata, in modo da approssimare il più possibile il k-mer profile reale.
	

%	Il programma effettua una regressione non lineare dei dati del k-mer profile, generando un nuovo profilo che cerca di approssimare il k-mer profile reale. Prendendo in input le letture del genoma che si vuole studiare, esso crea un modello che approssimi il più possibile il k-mer profile. La funzione $f(X)$ scelta per l'interpolazione delle frequenze dei k-mer trovati è la somma di quattro distribuzioni binomiali negative $\mathcal{NB}(X;p,n)$, rispettivamente per rappresentare k-mer eterozigoti trovati nel genoma diploide una volta (unici) o tre volte (duplicati), e k-mer omozigoti di cui si trovano due occorrenze (unici) o trovati quattro volte (duplicati). La funzione $f(X)$ è descritta dall'equazione~\vref{eqn:gnmscp_regression}, in cui $G$ rappresenta un coefficiente di scala legato alla dimensione del genoma, $\lambda$ e $\rho$ sono rispettivamente la media e la varianza della distribuzione. 
%	\begin{multline}
%		f(X) = G * (\alpha \mathcal{NB}(X;\lambda, \lambda/\rho) + \beta \mathcal{NB}(X;2\lambda, 2\lambda/\rho) + \\
%		\gamma \mathcal{NB}(X;3\lambda, 3\lambda/\rho) + \delta \mathcal{NB}(X;4\lambda, 4\lambda/\rho)  )	
%		\label{eqn:gnmscp_regression}
%	\end{multline}
%
%	I coefficienti $\alpha, \beta, \gamma$ e $\delta$ dipendono dai parametri $r$ e $d$, che rappresentano rispettivamente il rapporto di eterozigosi, cioè la percentuale di basi che sono specifiche a uno o due cromosomi omologhi, e la percentuale del genoma che è presente in due copie.
%	
%	Lo scopo del programma è quindi determinare i coefficienti $r, d, \lambda$ e $\rho$, oltre alla dimensione totale del genoma $G$. La funzione scelta $f(X)$, tramite cui poi può essere calcolata la dimensione del genoma, è quella che restituisce la minore somma dei quadrati degli errori residui (\textit{Residual Sum of Square Error} - \textit{RSSE}), cioè che minimizzi la somma tra i quadrati degli errori tra i valori osservati e quelli stimati, come descritto dall'equazione~\vref{eqn:gnmscp_RSSE}. Per dedurre i valori dei coefficienti, viene utilizzata la funzione \verb|nls| del linguaggio di programmazione R, che compie la regressione non lineare dei dati alla funzione obiettivo.
%	\begin{equation}
%		RSSE = \sum_{x=E}^{+\infty} \left(kmer_{obs}[x] - kmer_{pred}[x]\right)^2
%	\label{eqn:gnmscp_RSSE}
%	\end{equation}
%	Al termine, il programma mostra all'utente i dati relativi al genoma trovati, come il rapporto di eterozigosi, la media e la varianza della distribuzione, l'indice RSSE, che rappresenta la percentuale di k-mer non considerati dal modello, e la dimensione stimata del genoma.
%	
%	Eventuali errori di sequenziamento, ad esempio dovuti a duplicazioni con PCR o a sequenze contaminate, sono determinati solo empiricamente: dopo varie iterazioni del software in cui viene abbassata la soglia di copertura richiesta, i k-mer che non riescono ad essere rappresentati dal modello vengono identificati come errori di sequenziamento. 
%	
%	\begin{figure}
%		\centering
%		\includegraphics[width=0.6\textwidth]{capitoli/genomescope/gnmscp_genomescopeprofile.png}
%		\caption{TODO + TODO reference a figura.}
%		\label{fig:gnmscp_genomescopeprofile}
%	\end{figure}

	%TODO risultati?
	

	
	
	
\end{document}