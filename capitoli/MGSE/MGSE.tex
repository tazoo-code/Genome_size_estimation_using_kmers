\documentclass[crop=false, class=book]{standalone}

\usepackage{lipsum}

\begin{document}
	\section{MGSE}
	Il programma \textit{Mapping-based Genome Size Estimation} (\textit{MGSE}) stima la dimensione del genoma attraverso il calcolo della copertura media a partire dall'assembly ad alta contiguità delle letture~\cite{pucker2019MGSE}. Lo script è open-source e scritto in Python, e processa le informazioni sulla copertura delle letture di input restituendo la dimensione stimata del genoma.
	
	\subsection{Algoritmo}
	Posto che le letture di input siano distribuite equamente sull'intera sequenza del genoma, il programma ne stima la dimensione calcolandone la copertura media. Se infatti sono noti il numero $L$ di basi sequenziate e la copertura $C$ in una certa posizione, la lunghezza totale $N$ del genoma sarà pari a $N = L/C$.
	
	Dato che le letture non forniscono la stessa copertura in tutte le regioni, ad esempio a causa di sequenze ripetitive, per una stima reale della dimensione del genoma è importante un calcolo corretto della copertura media. Per questo, il programma permette all'utente di inserire una lista di regioni di riferimento usate per il calcolo della media e della mediana della copertura. A questo scopo può essere utile il programma \textit{Benchmarking Universal Single Copy Orthologs} (\textit{BUSCO})~\cite{simao2015busco}, che identifica un gruppo di sequenze \textit{bona fidae}, cioè che sono a singola copia e che hanno solo una sola corrispondenza all'interno del genoma~\cite{li2005trumatch}, le quali possono risultare adatte al calcolo della copertura media. 
	
	Il programma per poter effettuare la stima della dimensione del genoma riceve in input un file contenente la versione binaria compressa di sequenze allineate (file \textit{bam}~\cite{li2009sequence}), oppure un file contenente informazioni sulla copertura per ogni porzione della sequenza da sequenziare (file \textit{cov}~\cite{pucker2018genome}). Dopo aver calcolato la media e la mediana della copertura, il programma produce in output le dimensioni del genoma stimate per entrambi i valori.

	
	
	
\end{document}