\documentclass[crop=false, class=book]{standalone}

\usepackage{lipsum}

\begin{document}
	\section{MGSE}
	Il programma \textit{Mapping-based Genome Size Estimation} (\textit{MGSE}) stima la dimensione del genoma attraverso l'assembly delle letture utilizzando un genoma di riferimento ad alta contiguità \cite{pucker2019MGSE} TODO. Lo script è open-source e scritto in Python, e processa le informazioni sulla copertura delle letture di input restituendo la dimensione stimata del genoma.
	
	\subsection{Algoritmo}
	Posto che le letture di input siano distribuite equamente sull'intera sequenza del genoma, il programma ne stima la dimensione calcolandone la copertura media. Se infatti sono noti il numero $L$ di basi sequenziate e la copertura $C$ in una certa posizione, la lunghezza totale $N$ del genoma sarà pari a $N = L/C$.
	Dato che le letture sono 
	
	

	
	
	
\end{document}