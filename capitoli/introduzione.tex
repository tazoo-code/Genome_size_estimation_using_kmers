\documentclass[crop=false, class=book]{standalone}

\usepackage{lipsum}

\begin{document}
	\chapter{Introduzione}
	
	Il sequenziamento del DNA costituisce una tecnica fondamentale per lo studio del genoma di una specie, perché permette di determinare l'ordine delle basi azotate dei nucleotidi che costituiscono il DNA. Tale processo trova applicazione in molti studi biologici che riguardano vari ambiti, come ad esempio la medicina riproduttiva, l'oncologia o l'infettivologia, attraverso indagini tra cellule diverse dello stesso individuo o lo studio delle mutazioni genetiche tra individui di una stessa specie \cite{shendure2012expanding}. 
	
	% CONTENUTI (ordine da determinare)
	% a cosa serve (es: Assembly)
	% breve storia 
	% metodi (sperimentali vs computazionali) -> k-mer
	% cosa sono i k-mer
	% spiegare eterozigosi/omozigosi
	
	\section{Storia}
	Lo studio approfondito del DNA si sviluppa a partire dal 1953, con la scoperta della sua struttura tridimensionale ad opera di James Watson e Francis Crick \cite{watson1953molecular}, contribuendo all'analisi dell'azione degli acidi nucleici nella sintesi proteica. Solo nel 1977 però, vennero sviluppate le prime strategie sperimentali per il sequenziamento, come il famoso metodo Sanger \cite{sanger1977DNA, sanger1977nucleotide}
	
	%\lipsum[1]
	%\section{Lorem ipsum}
	%\lipsum[2]
	%\subsection{Dolor sit amet} 
	%\lipsum[3]
	%\subsubsection{Lorem ipsum}
	%\lipsum[3]
	
	% GLOSSARIO
	% omozigosi/eterozigosi
	% distribuzione binomiale negativa
	
	
	
	
\end{document}